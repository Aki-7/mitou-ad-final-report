\section{プロジェクト概要}

上記に掲げた目的のため開発面と社会実装の面で本プロジェクトでは以下のことを行った.

\subsection{開発面}

\begin{itemize}
      \item \textbf{Meta Quest 2 への対応} \\
            本プロジェクトの社会実装を進めるためには,利用可能なユーザの母数を増やすことが課題となる.
            2023年1月のSteamの調査
            \footnote{Steamハードウェア&ソフトウェア 調査: January 2023 (accessed on 11 Feb, 2023)}
            によると,VRヘッドセットの利用シェアの44.26\%Meta Quest 2であり,
            プロジェクト実施期間でMeta Quest 2に対応することで,ユーザの母数を増やすことができた.
            Meat Quest2に対応するため,本プロジェクトでは独自にネットワーク越しに
            レンダリングを行う仕組みを実装した.
      \item \textbf{デスクトップ環境の実装} \\
            本プロジェクトではXR Windowing Systemのユースケースをデスクトップ環境に絞った.
            他のユースケースとしては,もっと一般的に歩行中のユーザに様々な情報を提供するような
            システムなどが考えられるが,現状のヘッドセットの重量がまだまだ重い点や,
            既存の2Dアプリケーションというリソースが利用できる点から,デスクトップ環境を選択した.
            プロジェクト実施期間ではデスクトップ環境に必要な,アプリケーションランチャーや
            XR空間で2Dウィンドウをよりうまく利用するためのボードやシート面などの概念を定義し実装した.
      \item \textbf{2Dディスプレイ環境との統合} \\
            未踏アドバンスト事業実施以前のプロトタイプでは,XR空間のみを実装したが,
            XR空間のみではユーザが利用を開始するフローが確立できず,
            またヘッドセットを装着したまま長時間作業を続けることは負担が大きいため,
            プロジェクト実施期間では2Dディスプレイ環境とVR環境との両方のデスクトップ環境を実装し,
            その両者の間を行き来できるようにした.
\end{itemize}

\subsection{社会実装面}

\begin{itemize}
      \item \textbf{OSSプロジェクトとしての体裁整備} \\
            世間に認識されるOSSプロジェクトを立ち上げるには,単にソースコードを公開する
            だけでは不十分である.未踏アドバンスト事業期間ではOSSプロジェクトとして必要な,
            Webサイトの公開,LICENSEやCode of Conductの提示,
            DiscordサーバやTwitterアカウントなどのユーザとの窓口の整備,
            IssueやPull Requestのテンプレート作成,CI によるテスト自動化,
            開発者ドキュメントの作成などを行った.
            % 公開定例ミーティングはこれから
      \item \textbf{ファーストリリース} \\
            2023年1月16日にファーストリリースとして,プロジェクト外部のユーザが気軽に試せる環境を整え公開した.
            実際にユーザに試してもらいフィードバックを得るため,期間中にファーストリリースを行うことは必須条件であった.
            インストール手順を自動化するスクリプトの整備,Webサイトへの日本語・英語のインストールガイドの掲載,使用フローを解説する動画の制作,また各種メディアへのPR業務などを行った.
            Twitterなどのソーシャルメディアでは多くの反響を得ることができたほか,実際にインストールしたユーザも若干数現れ,期待した効果が得られた.
            % 反応とかのまとめ
      \item \textbf{商標登録の出願} \\
            リリースに伴い,プロジェクトとして商標を出願した.当初のプロジェクト名であった「ZIGEN」は商標登録の上で問題があったため,プロジェクトの名前を「Zwin」に変更した.ファーストリリースはZwinの名前のもと行った.
            % これからやる これに伴って名前をかえたことも。
      \item \textbf{寄付の募集} \\
            リリース後,ユーザからの寄付の申し出を受け,プロジェクトへの寄付の窓口を設置した.本報告書執筆時点で,設置から2週間で3名のユーザからスポンサーとして月額や単発の寄付を受けている.
            % 2週間で3人
\end{itemize}
