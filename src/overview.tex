\section{プロジェクト概要}

上記に掲げた目的のため開発面と社会実装の面で本プロジェクトでは以下のことを行った.

\subsection{開発面}

\begin{itemize}
      \item \textbf{Oculus Quest 2 への対応} \\
            本プロジェクトの社会実装を進めるためには,利用可能なユーザの母数を増やすことが課題となる.
            2023年1月のSteamの調査
            \footnote{Steamハードウェア&ソフトウェア 調査: January 2023 (accessed on 11 Feb, 2023)}
            によると,VRヘッドセットの利用シェアの44.26\%がOculus Quest 2であり,
            プロジェクト実施期間でOculus Quest 2に対応することで,ユーザの母数を増やすことができた.
            Oculus Quest2に対応するため,本プロジェクトでは独自にネットワーク越しに
            レンダリングを行う仕組みを実装した.
      \item \textbf{デスクトップ環境の実装}
            本プロジェクトではXR Windowing Systemのユースケースをデスクトップ環境に絞った.
            他のユースケースとしては,もっと一般的に歩行中のユーザに様々な情報を提供するような
            システムなどが考えられるが,現状のヘッドセットの重量がまだまだ重い点や,
            既存の2Dアプリケーションというリソースが利用できる点から,デスクトップ環境を選択した.
            プロジェクト実施期間ではデスクトップ環境に必要な,アプリケーションラウンチャーや
            XR空間で2Dウィンドウをよりうまく利用するためのボードやシート面などの概念を定義し実装した.
      \item \textbf{2Dディスプレイ環境との統合} \\
            未踏アドバンスト事業実施以前のプロトタイプでは,XR空間のみを実装したが,
            XR空間のみではユーザが利用を開始するフローが確立できず,
            また長時間使い続けることが難しいため,
            プロジェクト実施期間では2Dディスプレイ環境とVR環境との両方のデスクトップ環境を実装し,
            その両者の間を行き来できるようにした.
\end{itemize}

\subsection{社会実装面}

\begin{itemize}
      \item \textbf{OSSプロジェクトとしての体裁整備} \\
            世間に認識されるOSSプロジェクトを立ち上げるには,単にソースコードを公開する
            だけでは不十分である.本プロジェクト期間ではOSSプロジェクトとして必要な,
            ウェブサイトの公開,LICENSEやCode of Conductの提示,
            DiscordサーバやTwitterアカウントなどのユーザとの窓口の整備,
            IssueやPull Requestのテンプレート作成,CI によるテスト自動化,
            開発者ドキュメントの作成,公開定例ミーティングの開催,
            などを行った.
            % 公開定例ミーティングはこれから
      \item \textbf{ファーストリリース} \\
            % 反応とかのまとめ
      \item \textbf{商標登録の出願} \\
            % これからやる これに伴って名前をかえたことも。
      \item \textbf{寄付の募集} \\
            % 2週間で3人
\end{itemize}
