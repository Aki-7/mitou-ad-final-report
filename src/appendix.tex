\section{付録}

\subsection{用語説明}

\subsection*{OSS}
オープンソースソフトウェア(英:Open Source Software,略称:OSS)とは,
利用者の目的を問わずソースコードを使用,調査,再利用,修正,拡張,
再配布が可能なソフトウェアの総称である.

\subsubsection*{Window System}
ウィンドウシステム (Window System) とは,平行する複数のタスクにそれぞれ固有の領域として
ウィンドウを割当て,画面出力を多重化する,コンピュータ上のメカニズム,
及びそのためのソフトウェアのこと.
\footnote{https://ja.wikipedia.org/wiki/ウィンドウシステム より引用}

\subsubsection*{Wayland}
Wayland は,ディスプレイサーバとクライアント間の通信方法を記述した通信プロトコルである.
また,そのプロトコルをCで実装したライブラリでもある.
\footnote{https://ja.wikipedia.org/wiki/Wayland より引用}

\subsubsection*{OpenXR}
OpenXR(オープンエックスアール)は,バーチャルリアリティ及び拡張現実プラットフォーム
及びデバイスへのアクセスに関するオープンでロイヤリティフリーの規格である.
\footnote{https://ja.wikipedia.org/wiki/OpenXR より引用}

\subsubsection*{OpenGL}
OpenGL(オープンジーエル,英: Open Graphics Library)は,クロノス・グループ
(英: Khronos Group) が策定している,グラフィックスハードウェア向けの2次元/3次元
コンピュータグラフィックスライブラリである.
\footnote{https://ja.wikipedia.org/wiki/OpenGL より引用}

\subsection{関連Webサイト}

\textbf{Zwin project website}:https://zwin.dev

\textbf{Zwin GitHub organization}:https://github.com/zwin-project

\textbf{Wayland}:https://wayland.freedesktop.org

\textbf{OpenXR}:https://www.khronos.org/openxr

\textbf{OpenGL}:https://www.opengl.org
