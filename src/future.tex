\section{今後の課題,展望}

\subsection{開発面での課題}

未踏アドバンスト事業期間では最小限のデスクトップ環境の実装のみを行い,
ショートカットなどのより発展的なデスクトップ環境の整備までは行うことができなかった.
まずは自分達が日常生活で使い,ドッグフーディングできるためのデスクトップ環境に必要な機能を
開発していくつもりである.

また,Zwinがさらに発展してくためには,Zwin上で動くコンテンツをより増やしていく必要がある.
そのようなクリエイター向けの開発として,ゲームエンジンを用いたZwinアプリケーションの開発は
ぜひ進めていきたい.

\subsection{社会実装面での課題}

未踏アドバンスト事業期間ではユーザの関心を集めることができたが,
OSSとしてZwinが発展していく鍵はZwin上でアプリケーションを作成するクリエイターを
いかに増やすかであると考える.
Zwinアプリケーションを作成するハッカソンやクリエイター向けの定例ミーティングを開く
といったコミュニティ作りをしてくことで,Zwin開発者とクリエイター,そしてユーザからなる
エコシステムを実現していく必要があり,これが今後の課題となる.

\subsection{展望}

さらなる将来の展望として,Zwinによってデスクトップ環境のみならず,
視覚を拡張する情報システムにおいて,複数のアプリケーションを組み合わせるWindowing System
のパラダイムが浸透し,またZwinで検討されたデザインが活用されることを期待する.
これによってアプリケーションは空間全体ではなく,特定の機能だけを
うまく実現することが求められるようになる.
こうして様々な開発者や企業,グループがそれぞれの強みを生かしたアプリケーションを提供し,
参画していくことができるインクルーシブな環境を実現していきたい.
